\documentclass[a4paper,twoside,onecolumn,openany,article,10pt]{memoir}
\usepackage{xeCJK}
\usepackage{url}
\usepackage{hyperref}
\usepackage{amsmath}
\usepackage{amssymb}
\usepackage{amsthm}
\usepackage{enumerate}
%\usepackage{algorithm}
%\usepackage{algorithmicx}
%\usepackage{algpseudocode}
\usepackage{ascmac}
\usepackage{tikz}
\usepackage{ulem}
%\usepackage{stix}
%\usepackage{bm}
\defaultfontfeatures{Ligatures=TeX}

\setCJKmainfont[BoldFont=Noto Sans CJK JP Bold]{Noto Serif CJK JP}
%\setCJKmainfont{Noto Serif CJK JP}
\setCJKsansfont{Noto Sans CJK JP}
\setCJKmonofont{Noto Sans Mono CJK JP}

\newtheorem{theorem}{定理}
\theoremstyle{remark}
\newtheorem{remark}{\textbf{余談}}


\setmainfont[BoldFont=Noto Sans CJK JP Bold]{Noto Serif CJK JP}
\setsansfont{Noto Sans CJK JP}
\setmonofont{Inconsolata}

\usepackage{listings}

%\renewcommand{\algorithmcfname}{アルゴリズム}



\settrimmedsize{\stockheight}{\stockwidth}{*}

%\setlrmarginsandblock{1.5in}{1in}{*}
\setlrmarginsandblock{1.2in}{1.2in}{*}
\setulmarginsandblock{1.2in}{1.5in}{*}
\setheadfoot{20mm}{15mm}

%\newlength{\linespace}
%\setlength{\linespace}{\baselineskip}
%\setlength{\headheight}{\onelineskip}
%\setlength{\headsep}{\linespace}
%\addtolength{\headsep}{-\topskip}

%\setlength{\footskip}{\onelineskip}
%\setlength{\footnotesep}{\onelineskip}

\checkandfixthelayout

\counterwithout{section}{chapter}
\setsecnumdepth{subsubsection}


\title{アルゴリズムとデータ構造\\\vspace{.5em} \Large 分割統治法とソート}
\date{2018年6月22日}
\author{森~立平\\ \texttt{mori@c.titech.ac.jp}}

\begin{document}
\maketitle


\noindent
今日のメッセージ
\begin{itemize}
\item \textbf{分割統治法 = 「漸化式作ってそれをプログラムにするだけ」}
\item \textbf{ソートは分割統治法で解ける}
\item \textbf{分割統治法の時間計算量は漸化式を立てることによって見積もる}
\end{itemize}

\noindent
今日の目標
\begin{itemize}
\item いろいろなソートアルゴリズムを習得する
\end{itemize}




\section{分割統治法}
「アルゴリズム$\approx$漸化式」に従って漸化式をそのままアルゴリズムにしてしまう方法を「分割統治法(divide and conquer)」という。
ユークリッドの互除法や二分探索は分割統治法の特別な場合となる(漸化式の右辺に一つしか、今計算しようとしている関数が登場しないので「decrease and conquer」と呼ばれることもあるようだ)。
分割統治法が使える代表的な問題にソート問題がある。
\section{ソート問題}
与えられた数列 $a_1, a_2,\dotsc, a_n$ を小さい順に並び変える操作のことをソートと呼ぶ。
\begin{align*}
\text{入力}:&\quad a_1, a_2,\dotsc, a_n\\
\text{出力}:&\quad a_{i_1}, a_{i_2}, \dotsc, a_{i_n} \qquad \text{ここで $i_1,\dotsc,i_n\in\{1,2,\dotsc,n\}$ は互いに相異なり、$a_{i_j}\le a_{i_{j+1}}$ を満たす}
\end{align*}
このソートを計算するために分割統治法を使うことができる。
だが分割統治法を考える前に、素朴な方法でどれくらいの時間計算量があるかを考えてみよう。

\section{選択ソート}
\begin{equation*}
\mathrm{Ssort}(A) =
\begin{cases}
[\,],&\text{if } |A| = 0\\
[\min(A)] \circ \mathrm{Ssort}(A\setminus \min(A)),& \text{otherwise.}
\end{cases}
\end{equation*}
ここで$\circ$は配列の連結とする。

\begin{lstlisting}[basicstyle=\ttfamily\normalsize,showstringspaces=false,language=C,frame=single]
void Ssort(int A[], int n){
  int i, j, min;
  for(i = 0; i < n; i++){
    min = i;
    for(j = i+1; j < n; j++){
      if(A[j] < A[min]){
        min = j;
      }
    }
    int z = A[i];
    A[i] = A[min];
    A[min] = z;
  }
}
\end{lstlisting}

\section{挿入ソート}
\begin{equation*}
\mathrm{Isort}(A) =
\begin{cases}
[\,],&\text{if } |A| = 0\\
\mathrm{insert}(\mathrm{last}(A),\, \mathrm{Isort}(A - \mathrm{last}(A))),& \text{otherwise.}
\end{cases}
\end{equation*}
ここで
$\mathrm{insert}(a,\, B)$
は、整数$a$をソート済み配列$B$の適切な位置に挿入することで得られる配列である。

\begin{lstlisting}[basicstyle=\ttfamily\normalsize,showstringspaces=false,language=C,frame=single]
void Isort(int A[], int n){
  int i, j, k, z;
  for(i = 1; i < n; i++){
    for(j = 0; j < i && A[j] < A[i]; j++) ; //二分探索で置き換え可能
    z = A[i];
    for(k = i; k > j; k--) A[k] = A[k-1];
    A[j] = z;
  }
}
\end{lstlisting}


\section{マージソート}
マージソートは挿入ソートと良く似ているが、配列を半分のサイズに分割する。
\begin{align*}
\mathrm{Msort}([a_1,\dotsc,a_n]) =
\begin{cases}
[\,],& \text{if } n = 0\\
\mathrm{merge}(\mathrm{Msort}([a_1,\dotsc,a_{\lfloor n/2\rfloor}]),\, \mathrm{Msort}([a_{\lfloor n/2\rfloor+1}, \dotsc,a_n)]),& \text{otherwise.}
\end{cases}
\end{align*}
ここで$\mathrm{merge}(A, B)$は2つのソート済みの配列$A$と$B$について、$A\circ B$をソートした配列である。
\begin{lstlisting}[basicstyle=\ttfamily\normalsize,showstringspaces=false,language=C,frame=single]
int B[N];

void Msort(int A[], int n){
  int i, j, k;
  if(n <= 1) return;
  Msort(A, n/2);
  Msort(A+n/2, n - n/2);
  i = j = k = 0;
  while(i < n/2 && j < n - n/2){
    if(A[i] < A[n/2 + j]) B[k++] = A[i++];
    else B[k++] = A[n/2 + j++];
  }
  while(i < n/2) B[k++] = A[i++];
  while(j < n - n/2) B[k++] = A[n/2 + j++];
  for(i = 0; i < n; i++) A[i] = B[i];
}
\end{lstlisting}

\section{時間計算量の見積り}
\textbf{分割統治法の時間計算量は漸化式を立てる}ことによって見積もる。
簡単のために、$n$を2の羃と仮定するとマージソートの時間計算量$\chi(n)$は
\begin{align*}
\chi(n) &= 2 \chi(n/2) + c n
\end{align*}
を満たす。よってこの漸化式を解くと
\begin{align*}
\frac{\chi(n)}n &= \frac{\chi(n/2)}{n/2} + c = \chi(1) + c \log n
\end{align*}
となり$\chi(n) = O(n\log n)$であることが分かる。

%\section{クイックソート}

\section{比較に基づくソートの下界}
$T$回要素の比較をして、その結果だけを使って要素の置換をしてソート問題を解いたとしよう。
その場合、$2^T \ge n!$が成り立つ。$n! \ge \left(\frac{n}{e}\right)^n$ より、$T\ge n\log n - n \log e$ が得られる。
よって、比較に基づくソートでは$O(n\log n)$という時間計算量を改善することはできない。

\end{document}
