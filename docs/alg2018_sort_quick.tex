\documentclass[a4paper,twoside,onecolumn,openany,article,10pt]{memoir}
\usepackage{xeCJK}
\usepackage{url}
\usepackage{hyperref}
\usepackage{amsmath}
\usepackage{amssymb}
\usepackage{amsthm}
\usepackage{enumerate}
\usepackage{algorithm}
%\usepackage{algorithmicx}
\usepackage{algpseudocode}
%\usepackage{ascmac}
\usepackage{tikz}
\usepackage{ulem}
%\usepackage{stix}
%\usepackage{bm}
\defaultfontfeatures{Ligatures=TeX}

\setCJKmainfont[BoldFont=Noto Sans CJK JP Bold]{Noto Serif CJK JP}
%\setCJKmainfont{Noto Serif CJK JP}
\setCJKsansfont{Noto Sans CJK JP}
\setCJKmonofont{Noto Sans Mono CJK JP}

\newtheorem{theorem}{定理}
\newtheorem{lemma}{補題}
\theoremstyle{remark}
\newtheorem{remark}{\textbf{余談}}


\setmainfont[BoldFont=Noto Sans CJK JP Bold]{Noto Serif CJK JP}
\setsansfont{Noto Sans CJK JP}
\setmonofont{Inconsolata}

\usepackage{listings}

%\renewcommand{\algorithmcfname}{アルゴリズム}



\settrimmedsize{\stockheight}{\stockwidth}{*}

%\setlrmarginsandblock{1.5in}{1in}{*}
\setlrmarginsandblock{1.2in}{1.2in}{*}
\setulmarginsandblock{1.2in}{1.5in}{*}
\setheadfoot{20mm}{15mm}

%\newlength{\linespace}
%\setlength{\linespace}{\baselineskip}
%\setlength{\headheight}{\onelineskip}
%\setlength{\headsep}{\linespace}
%\addtolength{\headsep}{-\topskip}

%\setlength{\footskip}{\onelineskip}
%\setlength{\footnotesep}{\onelineskip}

\checkandfixthelayout

\counterwithout{section}{chapter}
\setsecnumdepth{subsubsection}


\title{アルゴリズムとデータ構造\\\vspace{.5em} \Large 分割統治法とソート}
\date{2018年6月26日}
\author{森~立平\\ \texttt{mori@c.titech.ac.jp}}

\begin{document}
\maketitle


\noindent
今日のメッセージ
\begin{itemize}
\item \textbf{分割統治法の時間計算量のは漸化式を解けば得られる}
\item \textbf{クイックソートはランダムにピボットを選べば平均$O(n\log n)$時間で計算できる}
\item \textbf{中央値の中央値アルゴリズムを使えば$O(n)$時間で$k$番目に小さい値が計算できる}
\end{itemize}

\noindent
今日の目標
\begin{itemize}
\item クイックソート、クイックセレクト、中央値の中央値アルゴリズムを理解する
\end{itemize}

\noindent
今日の演習の目標
\begin{itemize}
\item クイックソート、クイックセレクトなどのプログラムを書けるようになる
\end{itemize}

\noindent
今日の主な演習課題(提出締切は来週火曜日正午(12:00))
\begin{enumerate}
\item クイックソートとクイックセレクトのプログラムを書く
\end{enumerate}

\noindent
今日の演習時間のワークフロー
\begin{enumerate}
\item この資料をよく読み、\texttt{alg2018/sort} にある課題をやる
\end{enumerate}



\section{分割統治法の時間計算量}
分割統治法の時間計算量は漸化式を立てることによって見積る。
長さ$n$の配列に対するマージソートの時間計算量$\chi(n)$は
ある定数$c>0$について
\begin{align*}
\chi(n) &= \chi(\lfloor n/2\rfloor) + \chi(\lceil n/2\rceil) + c n
\end{align*}
を満たす。
\begin{lemma}
$\chi(n)$は$n$について単調非減少である。
\end{lemma}
\begin{proof}
明らかに$\chi(0)\le\chi(1)$が成り立つ。
$k\ge 1$とおく。
$n+1\le k$について$\chi(n)\le\chi(n+1)$が成り立つとするときに、$\chi(k)\le\chi(k+1)$を示せばよい。
$k$が偶数のとき、$\chi(k) = 2\chi(k/2) + c k$ と $\chi(k+1) = \chi(k/2) + \chi(k/2+1) + c(k+1)$が成り立つ。
帰納法の仮定より$\chi(k/2)\le\chi(k/2+1)$なので$\chi(k)\le\chi(k+1)$が成り立つ。
$k$が奇数のとき、$\chi(k) = \chi((k-1)/2) + \chi((k+1)/2) + c k$ と $\chi(k+1) = 2\chi((k+1)/2) + c(k+1)$が成り立つ。
帰納法の仮定より$\chi((k-1)/2)\le\chi((k+1)/2)$なので$\chi(k)\le\chi(k+1)$が成り立つ。
\end{proof}
よって$m=2^{\lceil \log n\rceil}$とおけば($m$は$n$を2羃に切り上げたもの)、$\chi(n)\le\chi(m)$が成り立つ。
$m$が2羃であることから$\chi(m)=O(n\log n)$が分かる。よって$\chi(n)=O(n\log n)$である。

長さ$n$のソート済み配列に対する二分探索の時間計算量$\chi(n)$は
ある定数$c>0$について
\begin{align*}
\chi(n) &\le \chi(\lceil n/2\rceil) + c
\end{align*}
を満たす。
マージソートの場合と同様に$\chi$の単調性が示せるので、$\chi(n)\le \chi(m)$であり、
\begin{align*}
\chi(m) &\le \chi(m/2) + c
\le \chi(1) + c \log m = \chi(1) + c \lceil \log n\rceil
\end{align*}
が得られる。よって$\chi(n)=O(\log n)$である。

\if0
一般的に分割統治法の時間計算量は次の形の不等式を満たす。
\begin{align*}
\chi(n) &\le a \chi(n/b) + T(n)
\end{align*}

まず簡単のため、$T(n)\le n$の場合を考えよう。
ある整数$t$について$n=b^t$と仮定して$C_k := \frac{\chi(b^k)}{b^k}$とおくと、
\begin{align*}
%C_k &\le \frac{a}{b} C_{k-1} + \frac{T(b^k)}{b^k}
C_k &\le \frac{a}{b} C_{k-1} + 1
% \le 1 + \frac{a}{b} + \left(\frac{a}{b}\right)^2 + \dotsb + \left(\frac{a}{b}\right)^k
\end{align*}
が成り立つ。よって
\begin{align*}
C_k - \frac{b}{b-a} &\le \frac{a}{b} \left(C_{k-1} - \frac{b}{b-a}\right) \le \left(\frac{a}{b}\right)^{k-1}\left(C_1 - \frac{b}{b-a}\right)
\end{align*}
が成り立ち、$\chi(b^t) = O(a^t + b^t)$、つまり$\chi(n) = O(n^{\log_b a} + n)$が得られる。
この時間計算量のオーダーは$T(n) = O(n^{\log_b a - \epsilon})$の場合についても正しい。
$b \ge a$の場合は$T(n)\le n$なので、$a\ge b$と仮定する。
\begin{align*}
C_k &\le \frac{a}{b} C_{k-1} + \frac{T(b^k)}{b^k}
\le \frac{a}{b} C_{k-1} + \frac{a^k}{b^k}
\end{align*}
という不等式が得られるが、
\begin{align*}
C_k &\le \frac{a}{b} C_{k-1} + \frac{T(b^t)}{b^t}
\end{align*}
\fi

\section{クイックソート}
クイックソートは次の漸化式に基づく。
\begin{align*}
\mathrm{Qsort}(A) =
\begin{cases}
[\,],& \text{if } |A| = 0\\
(B, C) := \mathrm{split}(A\setminus a_p, a_p),\, \mathrm{Qsort}(B) \circ \mathrm{Qsort}(C)
\end{cases}
\end{align*}
ここで、$a_p$は配列$A$の要素の一つであり、$\mathrm{split}(A, a)$は配列$A$を$a$以下の値からなる配列$B$と$a$より大きい値からなる配列$C$へ分割する関数である。
$a_p$の選択の仕方によって時間計算量は大きく変化する。

\section{クイックソートの時間計算量}
クイックソートの漸化式に現れる$a_p$のことをピボットと呼ぶ。ピボットとしては配列の中央値を選ぶのが最適である。
この章では配列に含まれる要素は全て異なる($\mathtt{A[i]\le A[j]}$かつ$\mathtt{A[i]\ge A[j]}$ならば$i = j$)と仮定する。
仮に中央値を$O(n)$時間で選択できるとすると、マージソートと同じ漸化式が得られるので時間計算量は$O(n\log n)$である。
仮に定数$\epsilon\in(0,1/2)$について$\epsilon n$番目以上$(1-\epsilon) n$番目未満の順番であるものをピボットとして$O(n)$時間で選択できるとすると
クイックソートの時間計算量は
\begin{align*}
\chi(n) = \chi(\epsilon n) + \chi((1-\epsilon)n) + cn
\end{align*}
となる(簡単のため切り捨てや切り上げの影響は無視することにする)。
多少天下り的ではあるが、ある定数$d>0$について帰納法で$\chi(n)\le dn\log n$を示す。
$\chi(k)\le dk\log k$ が $k< \lceil 1/\epsilon\rceil$に対して成り立つように$d>0$をとることができる。
このとき、$\chi(k)\le dk\log k$が$\lceil 1/\epsilon\rceil\le k<n$について成り立っているとすると,
\begin{align*}
\chi(n) &= \chi(\epsilon n) + \chi((1-\epsilon)n) + cn\\
&\le d\epsilon n\log(\epsilon n) + d(1-\epsilon)n\log((1-\epsilon)n) + cn\\
&= d n\log n - dn \left(-\epsilon \log\epsilon - (1-\epsilon)\log(1-\epsilon)\right) + c n
\end{align*}
ここでバイナリエントロピー関数$h(\epsilon):=-\epsilon\log\epsilon-(1-\epsilon)\log(1-\epsilon)$は$\epsilon\in(0,1)$について正である。
よって$d$を十分大きく取り直せば、$dh(\epsilon) \ge c$となる。
よって、$\chi(n)\le d n\log n$が得られる。よって、任意の自然数$n$について$\chi(n)\le dn\log n$である。

実際のクイックソートの実装ではピボットはランダムに選ぶことが一般的である。
ランダムに選べば高い確率で$\epsilon n$番目以上$(1-\epsilon) n$番目未満のピボットが選べるので、$O(n\log n)$時間で動作する。
一様な確率でピボットを選択するクイックソートにおける平均比較回数が$O(n\log n)$であることを示す。
ソートしたい配列に含まれる$i$番目に小さい要素を$a_i$とする。
$1\le i<j\le n$ について確率変数$X_{ij}$を
\begin{equation*}
X_{ij} := \mathbb{I}\left\{\text{クイックソートの中で$a_i$と$a_j$が比較される}\right\}
\end{equation*}
と定義する。
するとクイックソートの中で比較されるペアの個数は$X:=\sum_{1\le i<j\le n} X_{ij}$で表わされる。
\begin{align*}
\mathbb{E}[X] &=
\mathbb{E}\left[\sum_{1\le i<j\le n} X_{ij}\right]
=
\sum_{1\le i<j\le n} \mathbb{E}[X_{ij}]\\
&=
\sum_{1\le i<j\le n} \Pr\left(\text{クイックソートの中で$a_i$と$a_j$が比較される}\right)
\end{align*}
もしも$a_i$や$a_j$よりも先に$a_i < a_k < a_j$となる$a_k$がピボットとして選ばれてしまったとすると、
$a_i$と$a_j$が比較されることはない。すべての要素は一様な確率でピボットとして選ばれるので、
\begin{align*}
\Pr\left(\text{クイックソートの中で$a_i$と$a_j$が比較される}\right) = \frac2{j-i+1}
\end{align*}
となる。よって
\begin{align*}
\mathbb{E}[X] &=
\sum_{1\le i<j\le n} \frac2{j-i+1}\\
&= \sum_{i=1}^{n-1} \sum_{j=i+1}^n \frac2{j-i+1}\\
&= \sum_{i=1}^{n-1} \sum_{k=2}^{n-i+1} \frac2{k}\\
&\le \sum_{i=1}^{n-1} \sum_{k=1}^{n} \frac2{k}\\
&\le n\sum_{k=1}^{n} \frac2{k} = O(n\log n).
\end{align*}

\section{クイックセレクト}
配列の中で$k$番目に大きい要素を計算する問題を「選択問題」と呼ぶ。
選択問題はソートを用いれば$O(n\log n)$時間で解くことができるが、もっと効率のよい$O(n)$時間のアルゴリズムが存在する。
\begin{align*}
\mathrm{Qselect}(A, k) =
\begin{cases}
%[\,],& \text{if } |A| = 0\\
(B, C) := \mathrm{split}(A, a_p)\text{と置く}\\
a_p,& \text{if } |B| = k+1\\
\mathrm{Qselect}(B, k),&\text{if } |B| \le k\\
\mathrm{Qselect}(C, k-|B|),&\text{otherwise.}
\end{cases}
\end{align*}

\begin{lstlisting}[basicstyle=\ttfamily\normalsize,showstringspaces=false,language=C,frame=single]
#define N 10000000

int A[N];

/*
A[0], A[1], ..., A[n-1] の中でk+1番目に小さい値を返す関数。
n >= 1 && k >= 0 && k < n は保証されている。
ただし、Aの中身は並び換えてしまう。
*/
int quick_select(int A[], int n, int k){
  int i, j, pivot;

// 先頭の要素をピボットとする
  pivot = A[0];
  for(i = j = 1; i < n; i++){
    if(A[i] <= pivot){
      int z = A[j];
      A[j] = A[i];
      A[i] = z;
      j++;
    }
  }
  if(j == k+1) return pivot;
  else if(j < k+1) return quick_select(A+j, n-j, k-j);
  else return quick_select(A+1, j-1, k);
}
\end{lstlisting}

もしも$O(n)$時間で中央値が計算できたとすると、クイックセレクトの時間計算量$\chi(n)$は
$n$を2の羃とすると
\begin{equation*}
\chi(n) = \chi(n/2) + cn = c\left(n+\frac{n}2 + \frac{n}4 + \dotsb + 2\right) + \chi(1) \le 2c n + \chi(1) = O(n)
\end{equation*}
である。
クイックソートと同様に大体中央にあるようなピボットを選んだ場合も$O(n)$時間であることはすぐに分かる。


\section{中央値の中央値アルゴリズム}
乱数を用いないで決定的に$O(n)$時間で中央値を計算することもできる。
まず、$n$個の要素を5つづつ$\lceil n/5\rceil$個のグループに分ける。
各グループの中央値を計算して、中央値を集めて長さ$\lceil n/5\rceil$の配列を作る。
この長さ$\lceil n/5\rceil$の配列の中央値を再帰で計算し、その結果をピボットとして選択する。
アルゴリズムを以下に示す。


\begin{algorithm}
\floatname{algorithm}{アルゴリズム}
\caption{中央値の中央値アルゴリズムの擬似コード(入力: 整数の配列 $A$,非負の整数$k$.出力: 配列$A$の$k+1$番目に小さい要素.)}
\label{alg:qselectl}
\begin{algorithmic}
\If {配列$A$の長さが5以下}
  \State 解を計算して出力する.
\Else
  \State \textbf{配列$A$を長さ5づつに分割してそれぞれ中央値を計算し,長さ$\lceil n/5\rceil$の配列$A'$を作る.}
  \State \textbf{配列$A'$の中央値をピボットとして選択(再帰呼出).}
  \State 配列$A$のピボット以外の要素を「ピボット以下のもの」からなる配列$B$と「ピボットより大きいもの」からなる配列$C$の2つの配列に分割する.
  \State 配列$B$の要素の数を$r$とおく.
  \If {$r == k$}
    \State ピボットを解として出力する.
  \ElsIf {$r<k$}
    \State 配列$C$の中から$k-r$番目に小さい要素を解として出力する(再帰呼出).
  \Else
    \State 配列$B$の中から$k+1$番目に小さい要素を解として出力する(再帰呼出).
  \EndIf
\EndIf
\end{algorithmic}
\end{algorithm}

このアルゴリズムを「中央値の中央値アルゴリズム」と呼ぶ。中央値の中央値アルゴリズムの時間計算量$\chi(n)$は次の漸化式を満たす(切り捨て、切り上げの影響は無視することにする)。
\begin{align*}
\chi(n) = \chi(n/5) + \chi(7n/10) + cn
\end{align*}
ある定数$d>0$について$\chi(n)\le dn$を帰納法で示す。帰納法の仮定を用いると
\begin{align*}
\chi(n) \le \frac{d}{5}n + \frac{7d}{10}n + cn = \frac{9d + 10 c}{10} n.
\end{align*}
よって$d\ge 10 c$となるような$d$を選べば、$\chi(n)\le d n$を帰納法で示せる。

\section{比較に基づかないソートアルゴリズム: カウントソート、バケツソート、基数ソート}

\section{演習課題}
\begin{enumerate}
\item クイックセレクトのソースコードを参考にしてクイックソートのプログラムを書け。
\item クイックセレクト及びクイックソートのプログラムを改良して、重複した要素を持つ配列に対しても、それぞれ$O(n)$時間、$O(n\log n)$時間で動くプログラムを書け。ただし、ソートする配列以外に配列を使ってはならない。
\item {[発展的課題]} 中央値の中央値アルゴリズムのプログラムを書け。
\end{enumerate}

\end{document}
